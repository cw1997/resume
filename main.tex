% !TEX TS-program = xelatex
% !TEX encoding = UTF-8 Unicode
% !Mode:: "TeX:UTF-8"

\documentclass{resume}

\newcommand{\iconsection}[2]{
  \section[\texorpdfstring{#2}{#2}]{\faIcon{#1}\quad #2}
}

\begin{document}
% \pagenumbering{gobble} % suppress displaying page number
\sloppy

\name{昌维 (Wei Chang)}

% be careful of _ in emaill address
\contactInfo{WeChat/Telegram:changwei1006}{changwei1006@qq.com/gmail.com}{https://www.changwei.me}{} % {GitHub @cw1997}

\iconsection{user}{个人简介}
    本人是泛语言开发者;Node.js全栈工程师;从大三至今在两间公司实习长达5年;从初三学习编程至今长达13年;独自一人帮助学校开发并上架供全校数万师生使用的APP至GooglePlay和AppStore;同时掌握软硬件技术;熟悉原型图和UI设计工具;熟练撰写中英文产品说明文档和技术文档,习惯书写注释和语意清晰的git commit message;熟悉\LaTeX 和InDesign等排版技术;担任过台湾各类开源社群活动的志工以及学生会和社团干部;具备跨领域能力;具有大学助教和社团讲师经验(包括英文授课与实践教学);能够使用英语进行基本书面交流和简单口语表达(TOEIC 575分)
    \newline \textbf{现待业中,接受任何工作地点,主要擅长Node.js相关技术栈的前后端开发,但是也熟悉其他技术和语言,接受岗位调剂}

\iconsection{wrench}{技术能力}
    \begin{itemize}
      \item \textbf{前端}: JavaScript/TypeScript, HTML/CSS(Sass, TailwindCSS,styled-component), React.js(Next.js, Gatsby,js), Vue.js(Nuxt), jQuery, Webpack, Gulp.js, Rollup, Vite, Turborepo
      \item \textbf{后端}: Node.js(Fastify, Nest.js, tRPC, Express.js, Koa.js), PHP(Laravel, ThinkPHP), Python(Flask, fastapi), Java(Spring系列框架), Go, C/C++(了解STL,Socket API等), Shell(Bash), 汇编语言(X86, ARM, RISC-V, MIPS)
      \item \textbf{数据库}: Postgres(PL/pgSQL), MySQL, SQLite, SQL Server, Redis, MongoDB, ElasticSearch
      \item \textbf{运维}: 了解Linux基本运维知识,了解Proxmox VE/VMware EXSi等虚拟化平台,熟悉Docker,曾在学校机房见习,熟悉数据中心机房设备的操作。熟悉AWS,GCP,阿里云,腾讯云等公有云平台的运维操作
      \item \textbf{网络}: 了解OSI七层网络模型和TCP/IP协议栈,了解网络配置,了解使用Wireshark等工具分析协议栈
      \item \textbf{安全}: 熟悉XSS, CSRF, SQL Injection,逻辑缺陷等安全漏洞的机制,擅长防御式编程。曾经给百度BSRC,Wooyun,360补天等漏洞平台提交过多个知名平台的重大安全漏洞(包括百度贴吧,百度网盘等产品)
    \end{itemize}

\iconsection{graduation-cap}{教育背景}
    \datedsubsection{\textbf{台湾科技大学}|电子工程|\textit{在读硕士研究生}}{2021.9 \textasciitilde 2025.8}
        \ \textbf{光电半导体组}。主要研究基于PCB电路板,半导体元件和各种新型材料设计毫米波空间滤波器/天线。曾获校级奖学金
    \datedsubsection{\textbf{台湾科技大学}|电子工程|\textit{工学学士}}{2018.9 \textasciitilde 2021.8}
        \ \textbf{毕业设计:使用VerilogHDL在FPGA上构建RISC-V指令集架构的CPU}(\textit{https://github.com/risc-v-cpu})。
    熟悉计算机体系结构,操作系统原理,计算机网络,算法与数据结构,数据库,电子电路,MCU单片机(STM32, Arduino, 51单片机),FPGA(SystemVerilogHDL),PCB电路板设计,SMT贴片/THT插件元件焊接。曾获书卷奖(班级排名第一)
    \datedsubsection{\textbf{武汉船舶职业技术学院}| 软件技术| \textit{专科}}{2015.9 \textasciitilde 2018.8}
        \ \textbf{主要学习Java/PHP等Web前后端开发相关技术}。担任班级学习委员,负责班级助教工作。曾获校级奖学金

\iconsection{building}{实习经历}
    \datedsubsection{\textbf{PingCAP(北京平凯星辰科技发展有限公司)}| 前端开发}{北京,深圳,远程 | 2020 \textasciitilde 2024}
        \begin{itemize}
          \item https://tidb.net 从零开发TiDB中文社区官网,使用React.js和Next.js技术栈
          \item https://cn.pingcap.com 开发PingCAP公司新版官网,从旧版Hugo框架迁移至React.js和Gatsby.js技术栈
          \item https://pingcap.cn 开发平凯星辰公司官网,从零开始使用Next.js构建
        \end{itemize}

    \datedsubsection{\textbf{RisingWave Labs(北京奇点无限数据科技有限公司)}| 前端开发}{远程 | 2024 \textasciitilde 2025}
        \begin{itemize}
          \item 维护RisingWave公司旧版官网,对原有WordPress主题做二次开发,以及新开发WordPress插件与模板
          \item https://risingwave.com 开发RisingWave公司新版官网,从旧版WordPress迁移至React.js和Next.js技术栈
        \end{itemize}

\iconsection{list}{项目作品}
    \datedsubsection{\textbf{NTUST LIB}\textit{(跨平台APP)}}{https://library.ntust.edu.tw/p/405-1049-110462,c11171.php}
        \begin{itemize}
            \item \textbf{前端}基於React Native with Expo SDK開發,使用TypeScript程式語言。支援跨平台使用『iOS、Android 作業系統』,并且同时上架AppStore和GooglePlay,接入Firebase进行埋点分析,crash监控,接入Sentry收集日志
            \item \textbf{後端}基於Nest.js開發,使用TypeScript程式語言。使用MySQL作為Database,接入Sentry收集日志
            \item \textbf{技术栈}:React, React Native, Expo, Node.js, JavaScript, TypeScript, GitHub, Git, Github Actions, Linux, Docker, Nest.js, MySQL, Firebase, Sentry
        \end{itemize}
    \datedsubsection{\textbf{ez-react}\textit{(React.js复刻版)}}{https://ez-react.changwei.me , https://github.com/cw1997/ez-react}
        \begin{itemize}
            \item 使用TypeScript编写,我在参考官方React的工作原理后复刻了一个功能与API非常接近官方版React的框架
            \item \textbf{技术栈}:React, react-dom/client, JavaScript, TypeScript, Webpack
        \end{itemize}
    \datedsubsection{\textbf{ez-rtos}\textit{(嵌入式操作系统)}}{https://github.com/cw1997/ez-rtos}
        \begin{itemize}
            \item 使用ASM和C语言在STM32单片机上实现实时控制操作系统,专为ARM Cortex-M3内核设计
            \item 支持\textbf{任务切换}, \textbf{delay}, \textbf{内存分配器}, \textbf{关键区段}等操作系统的常见特性
            \item \textbf{技术栈}:C, ARM, Assembly Language, Keil, STM32, Cortex-M3
        \end{itemize}
    \datedsubsection{\textbf{inetutils}\textit{(网络协议栈)}}{https://github.com/cw1997/inetutils}
        \begin{itemize}
            \item 我参考GNU的inetutils功能后使用C语言复刻该工具,通过收发网络层的ICMP包实现了\texttt{ping}和\texttt{traceroute}
            \item \textbf{技术栈}:C, Socket API, GitHub Actions
        \end{itemize}
    \datedsubsection{\textbf{SDRAM Controller, SDRAM内存控制器}\textit{(计算机体系结构)}}{https://github.com/cw1997/SDRAM-Controller}
        \begin{itemize}
            \item 使用SystemVerilogHDL编写的SDRAM(Synchronous dynamic random-access memory)内存控制器,能够以正确的时序读写SDRAM内存颗粒并完成auto refresh等操作确保数据不丢失
            \item 本设计在基于 Altera Cyclone II EP2C35F672 芯片的 Terasic DE2(de2-35) FPGA 开发板上验证通过
            \item \textbf{技术栈}:FPGA, SystemVerilogHDL, VerilogHDL, SDRAM, Quartus II, ModelSim
        \end{itemize}

\iconsection{history}{其它经历}
    \datedsubsection{\textbf{台湾科技大学}| 助教}{}
        \begin{itemize}
            \item 负责课程\textbf{<電子工程總整>},教学内容包括\textbf{计算机网络TCP/IP协议栈,Raw Socket等基本概念与实作}
            \item 负责课程\textbf{<混合雲平台技術>},教学内容包括\textbf{在AWS配置云服务器,S3云存储,CDN,防火墙等,基本操作系统原理,Linux知识和Bash命令,安装架设网站所需要的各项服务组件,并且发布网站至公网}
        \end{itemize}
    \datedsubsection{\textbf{Google Developer Group on Campus}(谷歌开发者校园社群)| 讲师}{}
        \begin{itemize}
            \item \textbf{台湾科技大学}校区讲师,负责为社团成员\textbf{讲解前端开发和分享求职经验}
            \item \textbf{台湾大学}校区讲师,负责为社团成员\textbf{讲解SQL/Postgres/Supabase数据库和后端开发相关技术}
        \end{itemize}
    \datedsubsection{\textbf{台湾科技大学 学生会| 干部}}{}
        \begin{itemize}
            \item \textbf{行政部门}财务部部长,\textbf{负责会计工作,编列预决算报表}。学权部部员,\textbf{协助学校分发学生意见调查表单,协助办理校园演唱会等活动}。
            \item \textbf{立法部门(学生议会)}副议长,\textbf{负责协助议会开会事项,审理学生会行政部门的预决算报表,审核活动办理情况}
        \end{itemize}
    \datedsubsection{\textbf{武汉船舶职业技术学院 新媒体中心}| 部员}{}
        \begin{itemize}
            \item 负责培训其他部员,\textbf{讲授Adobe Premiere等剪辑软件,负责协助部员拍摄微电影,讲授Photoshop等设计软件,协助微信公众号与学校官方微博的发文排版等工作}
        \end{itemize}

\end{document}